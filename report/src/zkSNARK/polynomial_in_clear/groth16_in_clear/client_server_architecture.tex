\subsubsection{Back to client-server architecture}
So now if we go back to our problem where our client want to know if the server has some knowledge :
\\
\begin{tabular}{|c|c|}
  \hline
  \textbf{Client} & \textbf{Server} \TBstrut \\
  \hline
  Compute R & \TBstrut\\ 
  Setup(R) $\rightarrow (\sigma,\tau)$  & \TBstrut\\
  Send $\sigma$, $a_i$ inputs and R to the server & \TBstrut \\
  \hline
     & The server compute with his knowledge the $a_i$ outputs \TBstrut\\
     & and the witness \TBstrut\\
     & $Prove(R, \sigma, a_{i...l}, a_{l+1...m}) \rightarrow \pi$ \TBstrut\\
     & Send $\pi$ and $a_i$ outputs to the client \TBstrut\\
    \hline
    Run $Verify(R, \sigma, a_{i...l}, \pi) \rightarrow 0/1$ & \TBstrut\\

  \hline
\end{tabular}
\\
\\ If the ouput is 1 the client knows that the server knows the witness and the outputs send by the server are correct. But if it's 0 the client knows that the server doesn't know the secret or has calculated a wrong output value. By the way this protocol is zero-knowledge on the witness (i.e someone who intercept the communication will not learn anything about the witness).