\section{Protocol}
Now we have an idea of what are SNARK and how they work, our aim is to use them to create a remote polynomial evaluation. Where the server give a proof of the correctness of the computation. For this we'll use the groth16 algorithm implement in libsnark. All our results will be on this github \href{https://github.com/ruscot/snark-protocol-experiment}{repository}. We'll detail the steps of the Groth16 algorithm to have a better understanding of it.

First we have to convert our polynomial into a \textbf{rank 1 constraint system} (R1CS) which will be convert in a QAP and then we'll create our keys and start our protocol.

\newcommand\Aa{\textbf{Aa}}
\newcommand\Ba{\textbf{Ba}}
\newcommand\BaTwo{\textbf{Ba2}}
\newcommand\Ca{\textbf{Ca}}
\newcommand\As{\textbf{As}}
\newcommand\Bs{\textbf{Bs}}
\newcommand\BsTwo{\textbf{Bs2}}
\newcommand\Cs{\textbf{Cs}}
\newcommand\Hs{\textbf{Hs}}
\newcommand\Zs{\textbf{Zs}}
\newcommand\Za{\textbf{Za}}
\newcommand\AaLM{\textbf{AaLM}}
\newcommand\BaLM{\textbf{BaLM}}
\newcommand\CaLM{\textbf{CaLM}}
\newcommand\SOne{\textbf{S1}}
\newcommand\SOneOnC{\textbf{Sc}}
\newcommand\STwo{\textbf{S2}}
\newcommand\VOne{\textbf{V1}}
\newcommand\VTwo{\textbf{V2}}
\newcommand\SAnnexe{\textbf{Sa1}}
\newcommand\SSecondAnnexe{\textbf{Sb1}}
\newcommand\SThirdAnnexe{\textbf{Sc1}}

\newcommand\alphaOne{\textbf{$\alpha_1$}}
\newcommand\betaOne{\textbf{$\beta_1$}}
\newcommand\gammaOne{\textbf{$\gamma_1$}}
\newcommand\deltaOne{\textbf{$\delta_1$}}
\newcommand\betaTwo{\textbf{$\beta_2$}}
\newcommand\gammaTwo{\textbf{$\gamma_2$}}
\newcommand\deltaTwo{\textbf{$\delta_2$}}
\newcommand\CisDeltaOne{\textbf{CisDelta}}
\newcommand\CisDeltaStartOne{\textbf{CisDeltaStart}}
%\subsection{R1CS from Polynomials}
The aim of this section is to give an explanation with an example of how to construct a \textbf{rank-1 constraint system} (R1CS) from a polynomial.
\\ Let's say we have the polynomial $x^3+x+5$ and we want to construct a corresponding R1CS.
\subsubsection{Flattening}
The first step is a "flattening" procedure. We transform our original polynomial to a succession  of equation of type $x=y(op)z$.
\\Going back to our example we have :
\\$sym_1 = x*x$ (1)
\\$y=sym_1 *x$ (2)
\\$sym_2 = y+x$ (3)
\\$out=sym_2+5$ (4)
\subsubsection{From flattening to R1CS}
Now we can convert our succession of operation into a R1CS. An R1CS is a sequence of groups of three vectors (a,b,c) and the solution to an R1CS is a vector s where $sa * sb - sc = 0$
\\The length of each vector is equal to the total number of variables in the system, including a dummy variable "one" which represent the number 1.
\\So in our case the vector length is 7 which is : 
\\"one", "x", "out", "sym\_1", "y", "sym\_2"
\\And the assignement of each variable will correspond to one operation. So for our first operation we have :
\\$a=[0,1,0,0,0,0]$
\\$b=[0,1,0,0,0,0]$
\\$c=[0,0,0,1,0,0]$
\\I just skip the second one which is approximately the same as the first one, and for the third to perform the addition we have this :
\\$a=[0,1,0,0,1,0]$
\\$b=[1,0,0,0,0,0]$
\\$c=[0,0,0,0,0,1]$
\\And the last :
\\$a=[5,0,0,0,0,1]$
\\$b=[1,0,0,0,0,0]$
\\$c=[0,0,1,0,0,0]$
\\So the complete R1CS is :
\\A
\\$[0,1,0,0,0,0]$
\\$[0,0,0,1,0,0]$
\\$[0,1,0,0,1,0]$
\\$[5,0,0,0,0,1]$
\\B
\\$[0,1,0,0,0,0]$
\\$[0,1,0,0,0,0]$
\\$[1,0,0,0,0,0]$
\\$[1,0,0,0,0,0]$
\\C
\\$[0,0,0,1,0,0]$
\\$[0,0,0,0,1,0]$
\\$[0,0,0,0,0,1]$
\\$[0,0,1,0,0,0]$

\subsubsection{From R1CS to QAP}
Now we want to use our previous R1CS to convert it into a \textbf{quadratic arithmetic program} (QAP) form, which implement the same logic using polynomials. Such that if we evaluate the polynomials at x=1 then we get our first set of vectors, at x=2 the second and so forth.
\\To perform this transformation we will use the Lagrange Interpolation.
\\In our case we have 12 vectors of length 6. And we have to transform these into 6 groups of 3 polynomials, each one of degree 3. For example if we take the 1st column of A [0,0,0,5]. We can consider each element as the y-coordinate corresponding to x=1,2,3,4. So we get 4 sets of points (1,0), (2,0), (3,0), (4,5), it's an arbitrary interpretation we use to convert the R1CS into the QAP form. With lagrange interpolation we can find the polynomial passing through these 4 points.
\\It gives $0.833333333333333 * x^3 - 5.00000000000000 * x^2 + 9.16666666666667 * x - 5.00000000000000$
\\We just have to do this for all our points and then we have :
\\A polynomials
\\$[-5.0,9.166,-5.0,0.833]$
\\$[8.0,-11.333,5.0,-0.666]$
\\$[0.0,0.0,0.0,0.0]$
\\$[-6.0,9.5,-4.0,0.5]$
\\$[4.0,-7.0,3.5,-0.5]$
\\$[-1.0,1.833,-1.0,0.166]$
\\B polynomials
\\$[3.0,-5.166,2.5,-0.333]$
\\$[-2.0,5.166,-2.5,0.333]$
\\$[0.0,0.0,0.0,0.0]$
\\$[0.0,0.0,0.0,0.0]$
\\$[0.0,0.0,0.0,0.0]$
\\$[0.0,0.0,0.0,0.0]$
\\C polynomials
\\$[0.0,0.0,0.0,0.0]$
\\$[0.0,0.0,0.0,0.0]$
\\$[-1.0,1.833,-1.0,0.166]$
\\$[4.0,-4.333,1.5,-0.166]$
\\$[-6.0,9.5,-4.0,0.5]$
\\$[4.0,-7.0,3.5,-0.5]$
\\And then to finish we have to defined Z as $(x-1)(x-2)(x-3)...$ the polynomials that is equal to zero at all points corresponding to logic gates. So in our case Z=(x-1)(x-2)(x-3)(x-4)
\\For a more complete explanation see \cite{QAP_creation} and \cite{QAP_creation_2}

Now we'll give an explanation of the Groth16 algorithm.
\subsection{Snark protocol in clear}
If the client has a QAP with the polynomials $\{A_i[X]$, $B_i[X]$, $C_i[X]\}_{i=0}^n$ where he knows the value $a_{1...l,l+1...m}$ to solve it. Polynomials $A_i$, $B_i$ and $C_i$ are of degree n and Z is of degree n. 
\\ Let's call the QAP \textbf{R} such that  $R = \{F, m, l, \{A_i, B_i, C_i\}_{i=0}^m, \{Z_i\}_{i=0}^n\}$
\\ We define 3 methods Setup, Prove and Verify such that :
\\$Setup(R) \rightarrow (\sigma,\tau)$ 
\\$Prove(R, \sigma, a_{i...l}, a_{l+1...m}) \rightarrow \pi$
\\$Verify(R, \sigma, a_{i...l}, \pi) \rightarrow 0/1$

\subsubsection{Setup function}
\underline{Setup(R) :}
\tabNormal $\alpha \xleftarrow[]{\$} F^*$
\tabNormal $\beta \xleftarrow[]{\$} F^*$
\tabNormal $\gamma \xleftarrow[]{\$} F^*$
\tabNormal $\delta \xleftarrow[]{\$} F^*$ 
\tabNormal $s \xleftarrow[]{\$} F^*$
\tabNormal $\tau = (\alpha, \beta,\gamma,\delta,s)$
\tabNormal $\alphaOne = [\alpha]_1$
\tabNormal $\betaOne = [\beta]_1$
\tabNormal $\gammaOne = [\gamma]_1$
\tabNormal $\deltaOne = [\delta]_1$
\tabNormal $\SOne=\{[s^i]_1\}_{i=0}^{n-1}$
\tabNormal $\SAnnexe = \{[\frac{\beta A_i (s) + \alpha B_i (s) + C_i (s)}{\delta}]_1\}_{i=l+1}^m$
\tabNormal $\SSecondAnnexe = \{[\frac{s^i Z_i(s)}{\delta}]_1\}_{i=0}^{n-1}$
\tabNormal $\SThirdAnnexe = \{[\frac{\beta A_i (s) + \alpha B_i (s) + C_i (s)}{\gamma}]_1\}_{i=0}^l$
\tabNormal $\sigma1=(\alphaOne, \betaOne,\gammaOne,\deltaOne, \SOne, \SAnnexe , \SSecondAnnexe,\SThirdAnnexe, g, G_1,p)$
\tabNormal $\betaTwo = [\beta]_2$
\tabNormal $\gammaTwo = [\gamma]_2$
\tabNormal $\deltaTwo = [\delta]_2$
\tabNormal $\STwo = \{[s^i]_2\}_{i=0}^{n-1}$
\tabNormal $\sigma2=(\betaTwo,\gammaTwo,\deltaTwo, \STwo, h, G_2,p)$
\tabNormal $\sigma = (\sigma1, \sigma2, A_i, B_i, C_i)$
\tabNormal return ($\sigma, \tau$)

Only $\sigma$ is sent to the person who want to prove something.

\paragraph{Prove function}
We have $a_0 = 1$ it's a constant due to the R1CS constraint. \\
\underline{Prove(crs, $a_{1..l}$, $a_{l+1..m}$) :}
\tabNormal $r \xleftarrow[]{\$}F*$
\tabNormal $k \xleftarrow[]{\$}F*$
\tabNormal $\Aa =\sum_{i=0}^m a_i A_i$
\tabNormal $\Ba=\sum_{i=0}^m a_i B_i$
\tabNormal $\Ca =\sum_{i=0}^m a_i C_i$
\tabNormal U = $\alphaOne \Pi_{i=0}^{m} \SOne_i^{\Aa_i} (\deltaOne)^r $
\tabNormal \VOne = $\betaOne \Pi_{i=0}^{m} \SOne_i^{\Ba_i} (\deltaOne)^k $
\tabNormal \VTwo = $\betaTwo \Pi_{i=0}^{m} \STwo_i^{\Ba_i} (\deltaTwo)^k $
\tabNormal We compute H(s) such that  \[ (\sum_{i=0}^m a_i A_i(s)) (\sum_{i=0}^m a_i B_i(s)) = \sum_{i=0}^m a_i C_i(s) + H(s) Z(s) \]
\tabNormal $S=\Pi_{i=l+1}^{m}(\SAnnexe_i)^{a_i} \CisDeltaOne_i$
\tabNormal W = $\frac{(\Pi_{i=0}^n S_i) (\Pi_{i=0}^n (\SSecondAnnexe_i)^{H_i}) U^k \VOne^k}{\delta1^{rk}} $
\tabNormal $\pi=(U, W, \VTwo)$ 
\tabNormal return $\pi=(U, W, \VTwo)$, $a_l$ // the one corresponding to the output.


\subsubsection{Verify function}
\underline{$Verify(R, \sigma, a_{i...l}, \pi)$ :}
\tabNormal $\Aa = \sum_{i=0}^{l} a_i A_i$
\tabNormal $\Ba=\sum_{i=0}^{l} a_i B_i$
\tabNormal $\Ca =\sum_{i=0}^{l} a_i C_i$
\tabNormal $Y=\frac{\beta \Aa (s) + \alpha \Ba (s) + \Ca (s)}{\gamma}$
\tabNormal if $UV == \alpha \beta + Y \gamma +  W\delta $ 
\tabOne return 1
\tabNormal else
\tabOne return 0 \\
See \cite{On_the_Size_of_Pairing_based_Non_interactive_Arguments} for more explanations.

\subsubsection{Proof of the equation}
\label{sec:EquationProofNotInZK} 
What we want is to check the following equality : $(\sum_{i=0}^m a_i A_i(s)) (\sum_{i=0}^m a_i B_i(s)) = \sum_{i=0}^m a_i C_i(s) + H(s) Z(s)$
\\Detailled calculation for the if statement in verify function :
\\ UV = $(\alpha + \sum_{i=0}^{m} a_i A_i (s) + r\delta ) (\beta + \sum_{i=0}^{m} a_i B_i (s) + k\delta )$
\\
\\ = $\alpha\beta + \alpha(\sum_{i=0}^{m} a_i B_i (s)) + k\alpha\delta + \beta(\sum_{i=0}^{m} a_i A_i (s))+ (\sum_{i=0}^{m} a_i A_i (s))(\sum_{i=0}^{m} a_i B_i (s)) + k\delta(\sum_{i=0}^{m} a_i A_i (s)) + r\delta\beta + r\delta(\sum_{i=0}^{m} a_i B_i (s)+ rk\delta\delta$
\\
\\
\\ $\alpha \beta + Y \gamma +  W\delta$ = $\alpha \beta + (\frac{\beta \sum_{i=0}^{l} a_i A_i (s) + \alpha \sum_{i=0}^{l} a_i B_i(s) + \sum_{i=0}^{l} a_i C_i(s) }{\gamma}) \gamma  + (S + \frac{H(s)Z(s)}{\delta} + Uk + rV - rk\delta) \delta$
\\
\\ = $\alpha \beta + \beta \sum_{i=0}^{l} a_i A_i (s) + \alpha \sum_{i=0}^{l} a_i B_i(s) + \sum_{i=0}^{l} a_i C_i(s)  + (S + Uk + rV - rk\delta) \delta + H(s)Z(s)$
\\
\\ = $\alpha \beta + \beta \sum_{i=0}^{l} a_i A_i (s) + \alpha \sum_{i=0}^{l} a_i B_i(s) + \sum_{i=0}^{l} a_i C_i(s)  + (\frac{\beta \sum_{i=l+1}^m a_i A_i(s) + \alpha \sum_{i=l+1}^m a_i B_i(s) + \sum_{i=l+1}^m a_i C_i(s)}{\delta} + (\alpha + \sum_{i=0}^{m} a_i A_i (s) + r\delta)k + r(\beta + \sum_{i=0}^{m} a_i B_i (s) + k\delta) - rk\delta) \delta + H(s)Z(s)$
\\
\\ = $\alpha \beta + \beta \sum_{i=0}^{l} a_i A_i (s) + \alpha \sum_{i=0}^{l} a_i B_i(s) + \sum_{i=0}^{l} a_i C_i(s)  + \beta \sum_{i=l+1}^m a_i A_i(s) + \alpha \sum_{i=l+1}^m a_i B_i(s) + \sum_{i=l+1}^m a_i C_i(s) + ((\alpha + \sum_{i=0}^{m} a_i A_i (s) + r\delta)k + r(\beta + \sum_{i=0}^{m} a_i B_i (s) + k\delta) - rk\delta) \delta + H(s)Z(s)$
\\
\\ = $\alpha \beta + \beta \sum_{i=0}^{m} a_i A_i (s) + \alpha \sum_{i=0}^{m} a_i B_i(s) + \sum_{i=0}^{m} a_i C_i(s) + ((\alpha + \sum_{i=0}^{m} a_i A_i (s) + r\delta)k + r(\beta + \sum_{i=0}^{m} a_i B_i (s) + k\delta) - rk\delta) \delta + H(s)Z(s)$
\\
\\ = $\alpha \beta + \beta \sum_{i=0}^{m} a_i A_i (s) + \alpha \sum_{i=0}^{m} a_i B_i(s) + \sum_{i=0}^{m} a_i C_i(s) + (k \alpha + k \sum_{i=0}^{m} a_i A_i (s) + k r\delta + r \beta + r \sum_{i=0}^{m} a_i B_i (s) + r k\delta - rk\delta) \delta + H(s)Z(s)$
\\
\\ = $\alpha\beta + \beta\sum_{i=0}^{m} a_i A_i (s) + \alpha\sum_{i=0}^{m} a_i B_i(s) + \sum_{i=0}^{m} a_i C_i(s) + k\delta\alpha + k\delta\sum_{i=0}^{m} a_i A_i (s) + kr\delta\delta + r\delta\beta + r\delta\sum_{i=0}^{m} a_i B_i (s) + H(s)Z(s)$
\\
\\
\\ \texthl{$\alpha\beta$} $ + $ \texthl{$\alpha(\sum_{i=0}^{m} a_i B_i (s))$} $ + $ \texthl{$k\alpha\delta$} $ + $ \texthl{$\beta(\sum_{i=0}^{m} a_i A_i (s))$} $+ (\sum_{i=0}^{m} a_i A_i (s))(\sum_{i=0}^{m} a_i B_i (s)) + $ \texthl{$k\delta(\sum_{i=0}^{m} a_i A_i (s))$} $+ $ \texthl{$r\delta\beta$} $ + $ \texthl{$r\delta(\sum_{i=0}^{m} a_i B_i (s)$} $+ $ \texthl{$rk\delta\delta$} \textbf{\large=} \texthl{$\alpha\beta$} $ + $ \texthl{$\beta\sum_{i=0}^{m} a_i A_i (s)$} $ + $ \texthl{$\alpha\sum_{i=0}^{m} a_i B_i(s)$} $ + \sum_{i=0}^{m} a_i C_i(s) + $ \texthl{$k\delta\alpha$} $ + $\texthl{$k\delta\sum_{i=0}^{m} a_i A_i(s)$} $ + $ \texthl{$kr\delta\delta$} $+ $\texthl{$r\delta\beta$}$ + $\texthl{$r\delta\sum_{i=0}^{m} a_i B_i(s)$} $ + H(s)Z(s)$
\\
\\ $\Leftrightarrow (\sum_{i=0}^{m} a_i A_i (s))(\sum_{i=0}^{m} a_i B_i (s))$ \textbf{\large=} $\sum_{i=0}^{m} a_i C_i(s) + H(s)Z(s)$

\subsubsection{Back to client-server architecture}
So now if we go back to our problem where our client want to know if the server has some knowledge :
\\
\begin{tabular}{|c|c|}
  \hline
  \textbf{Client} & \textbf{Server} \TBstrut \\
  \hline
  Compute R & \TBstrut\\ 
  Setup(R) $\rightarrow (\sigma,\tau)$  & \TBstrut\\
  Send $\sigma$, $a_i$ inputs and R to the server & \TBstrut \\
  \hline
     & The server compute with his knowledge the $a_i$ outputs \TBstrut\\
     & and the witness \TBstrut\\
     & $Prove(R, \sigma, a_{i...l}, a_{l+1...m}) \rightarrow \pi$ \TBstrut\\
     & Send $\pi$ and $a_i$ outputs to the client \TBstrut\\
    \hline
    Run $Verify(R, \sigma, a_{i...l}, \pi) \rightarrow 0/1$ & \TBstrut\\

  \hline
\end{tabular}
\\
\\ If the ouput is 1 the client knows that the server knows the witness and the outputs send by the server are correct. But if it's 0 the client knows that the server doesn't know the secret or has calculated a wrong output value. By the way this protocol is zero-knowledge on the witness (i.e someone who intercept the communication will not learn anything about the witness).

\subsubsection{Problem}
\label{sec:ProblemWithoutZK}
But we have a problem with these scheme. Since the server knows $\alpha, \beta,\delta,x$. He can cheat by this way :
\\Pick U,V over F at random
\\Compute W=$\frac{UV-\alpha\beta-\sum_{i=0}^l(a_i(\beta A_i(s) + \alpha B_i(s) + C_i(s)))}{\delta}$
\\Now our equality is :
\\$UV = \alpha \beta + \frac{\beta \sum_{i=0}^{l} a_i A_i(s) + \alpha \sum_{i=0}^{l} a_i B_i(s) + \sum_{i=0}^{l} a_i C_i(s)}{\gamma} \gamma +  \frac{UV-\alpha\beta-\sum_{i=0}^l(a_i(\beta A_i(x) + \alpha B_i(x) + C_i(x)))}{\delta}\delta $
\\
\\$UV = \alpha \beta + \beta \sum_{i=0}^{l} a_i A_i(s) + \alpha \sum_{i=0}^{l} a_i B_i(s) + \sum_{i=0}^{l} a_i C_i(s) +  UV-\alpha\beta-\sum_{i=0}^l(a_i(\beta A_i(s) + \alpha B_i(s) + C_i(s)))$
\\
\\$UV = \alpha \beta + \beta \sum_{i=0}^{l} a_i A_i(s) + \alpha \sum_{i=0}^{l} a_i B_i(s) + \sum_{i=0}^{l} a_i C_i(s) +  UV-\alpha\beta-\sum_{i=0}^l(a_i(\beta A_i(s) + \alpha B_i(s) + C_i(s)))$
\\
\\$0 = 0$
So the if statement in the verify function will return true even if the person who compute the proof didn't know the $a_{l+1..m}$ (the witness).
\newpage
\subsection{Snark protocol in ZK}
Now we have our protocol and the idea behind that. We want to remove the problem identify above. To perform this we'll use pairing with elliptic curves. $G1*G2 \rightarrow Gt$
\\Now let's define R like $R = \{p, G1, G2, Gt, e, g, h, l, A, B, C, Z\}$. With $(p, G1, G2, Gt, e, g, h)$ a bilinear map, with the following definition :

\textbf{Bilinear map} is defined by seven elements $(p, G_1, G_2, G_T, e, g, h)$ such that :
\tabNormal - $e:G_1*G_2\rightarrow G_T$
\tabNormal - $G_1$, $G_2$, $G_T$ are groups of prime order p
\tabNormal - g is a generator of $G_1$
\tabNormal - h is a generator of $G_2$
\tabNormal - e(g,h) is a generator of $G_T$
\tabNormal - $e(g^a, h^b) = e(g,h)^{ab}$ \\
\\For the same method Setup, Prove and Verify we just change the value of R and their output.

\subsubsection{Setup function}
\underline{Setup(R) :}
\tabNormal $\alpha \xleftarrow[]{\$} F^*$
\tabNormal $\beta \xleftarrow[]{\$} F^*$
\tabNormal $\gamma \xleftarrow[]{\$} F^*$
\tabNormal $\delta \xleftarrow[]{\$} F^*$ 
\tabNormal $s \xleftarrow[]{\$} F^*$
\tabNormal $\tau = (\alpha, \beta,\gamma,\delta,s)$
\tabNormal $\alphaOne = [\alpha]_1$
\tabNormal $\betaOne = [\beta]_1$
\tabNormal $\gammaOne = [\gamma]_1$
\tabNormal $\deltaOne = [\delta]_1$
\tabNormal $\SOne=\{[s^i]_1\}_{i=0}^{n-1}$
\tabNormal $\SAnnexe = \{[\frac{\beta A_i (s) + \alpha B_i (s) + C_i (s)}{\delta}]_1\}_{i=l+1}^m$
\tabNormal $\SSecondAnnexe = \{[\frac{s^i Z_i(s)}{\delta}]_1\}_{i=0}^{n-1}$
\tabNormal $\SThirdAnnexe = \{[\frac{\beta A_i (s) + \alpha B_i (s) + C_i (s)}{\gamma}]_1\}_{i=0}^l$
\tabNormal $\sigma1=(\alphaOne, \betaOne,\gammaOne,\deltaOne, \SOne, \SAnnexe , \SSecondAnnexe,\SThirdAnnexe, g, G_1,p)$
\tabNormal $\betaTwo = [\beta]_2$
\tabNormal $\gammaTwo = [\gamma]_2$
\tabNormal $\deltaTwo = [\delta]_2$
\tabNormal $\STwo = \{[s^i]_2\}_{i=0}^{n-1}$
\tabNormal $\sigma2=(\betaTwo,\gammaTwo,\deltaTwo, \STwo, h, G_2,p)$
\tabNormal $\sigma = (\sigma1, \sigma2, A_i, B_i, C_i)$
\tabNormal return ($\sigma, \tau$)

Only $\sigma$ is sent to the person who want to prove something.

\paragraph{Prove function}
We have $a_0 = 1$ it's a constant due to the R1CS constraint. \\
\underline{Prove(crs, $a_{1..l}$, $a_{l+1..m}$) :}
\tabNormal $r \xleftarrow[]{\$}F*$
\tabNormal $k \xleftarrow[]{\$}F*$
\tabNormal $\Aa =\sum_{i=0}^m a_i A_i$
\tabNormal $\Ba=\sum_{i=0}^m a_i B_i$
\tabNormal $\Ca =\sum_{i=0}^m a_i C_i$
\tabNormal U = $\alphaOne \Pi_{i=0}^{m} \SOne_i^{\Aa_i} (\deltaOne)^r $
\tabNormal \VOne = $\betaOne \Pi_{i=0}^{m} \SOne_i^{\Ba_i} (\deltaOne)^k $
\tabNormal \VTwo = $\betaTwo \Pi_{i=0}^{m} \STwo_i^{\Ba_i} (\deltaTwo)^k $
\tabNormal We compute H(s) such that  \[ (\sum_{i=0}^m a_i A_i(s)) (\sum_{i=0}^m a_i B_i(s)) = \sum_{i=0}^m a_i C_i(s) + H(s) Z(s) \]
\tabNormal $S=\Pi_{i=l+1}^{m}(\SAnnexe_i)^{a_i} \CisDeltaOne_i$
\tabNormal W = $\frac{(\Pi_{i=0}^n S_i) (\Pi_{i=0}^n (\SSecondAnnexe_i)^{H_i}) U^k \VOne^k}{\delta1^{rk}} $
\tabNormal $\pi=(U, W, \VTwo)$ 
\tabNormal return $\pi=(U, W, \VTwo)$, $a_l$ // the one corresponding to the output.


\subsubsection{Verify function}
\underline{$Verify(R, \sigma, a_{i...l}, \pi)$ :}
\tabNormal $\Aa = \sum_{i=0}^{l} a_i A_i$
\tabNormal $\Ba=\sum_{i=0}^{l} a_i B_i$
\tabNormal $\Ca =\sum_{i=0}^{l} a_i C_i$
\tabNormal $Y=\frac{\beta \Aa (s) + \alpha \Ba (s) + \Ca (s)}{\gamma}$
\tabNormal if $UV == \alpha \beta + Y \gamma +  W\delta $ 
\tabOne return 1
\tabNormal else
\tabOne return 0 \\
See \cite{On_the_Size_of_Pairing_based_Non_interactive_Arguments} for more explanations.

\subsubsection{Proof of the equality}
What we want is to check the following equality :
\[ (\sum_{i=0}^m a_i A_i(s)) (\sum_{i=0}^m a_i B_i(s)) = \sum_{i=0}^m a_i C_i(s) + H(s) Z(s) \]
Here is the detailled calcul for the if statement in verify :
\\ $e(U,\VTwo) = e((\alphaOne \Pi_{i=0}^{m} \SOne_i^{\Aa_i} (\deltaOne)^r), (\betaTwo \Pi_{i=0}^{m} \STwo_i^{\Ba_i} (\deltaTwo)^k)) $
\\
\\ = $ [(\alpha + \sum_{i=0}^{m} a_i A_i (s) + r\delta) (\beta + \sum_{i=0}^{m} a_i B_i (s) + k\delta)]_T$
\\ 
\\ $e(\alphaOne, \betaTwo) e(Y, \gammaTwo)  e(W, \deltaTwo) = e(\alphaOne, \betaTwo) e(\Pi_{i=0}^{l}(\SThirdAnnexe_i)^{a_i}, \gammaTwo) e(\frac{S (\Pi_{i=0}^n (\SSecondAnnexe_i)^{H_i}) U^k \VOne^k}{\deltaOne^{rk}}, \deltaTwo)$
\\
\\ = $[\alpha \beta]_T [(\frac{\beta \sum_{i=0}^{l} a_i A_i (s) + \alpha \sum_{i=0}^{l} a_i B_i(s) + \sum_{i=0}^{l} a_i C_i(s) }{\gamma}) \gamma]_T  [(S + \frac{H(s)Z(s)}{\delta} + Uk + rV - rk\delta) \delta]_T$
\\
\\= $[\alpha \beta + (\frac{\beta \sum_{i=0}^{l} a_i A_i (s) + \alpha \sum_{i=0}^{l} a_i B_i(s) + \sum_{i=0}^{l} a_i C_i(s) }{\gamma}) \gamma + (S + \frac{H(s)Z(s)}{\delta} + Uk + rV - rk\delta) \delta]_T$
\\\\For our both side we have the same equation as the proof in \hyperref[sec:EquationProofNotInZK]{4.4} just in the $g_T$ exponent.

\subsubsection{Good point}
We have solved our problem \hyperref[sec:ProblemWithoutZK]{above}, with the encryption of $\alpha, \beta,\gamma,s$ the server can't compute W as before :
\\W=$\frac{UV-\alpha\beta-\sum_{i=0}^l(a_i(\beta A_i(s) + \alpha B_i(s) + C_i(s)))}{\delta}$
\\
or he has break the descrete logarithm problem in order to find s,$\beta, \alpha$ or $\gamma$.
\newpage
\subsection{Go back to our problem}
Now we have this protocol with zkSNARKs we want something quite similar. Back to a connexion client-server the problem is the following. A client want to evaluate a polynomial on a point, but he want that the server achieve this and gave him a proof of the correctness of the result.
\\We have something like this :
\\

\begin{tabular}{|c|c|}
  \hline
  \textbf{Client} & \textbf{Server} \TBstrut \\
  \hline
  P$\in F[X]$ a polynomial, we want to evaluate it on x $\in F$ & \TBstrut \\ 
  Send P and x to the server & \TBstrut \\
  \hline
     & Compute y = P(x) \TBstrut \\
     & Compute $\pi$ a proof that y is correct \TBstrut \\
     & Send $\pi$ and y to the client \TBstrut \\
    \hline
    With $\pi$ anyone can check that y is correct & \TBstrut \\
  \hline
\end{tabular}
\\
\\
In order to compute the proof of our computation we will use the SNARK protocol with QAP explained above.

\subsubsection{Protocol in clear}
\begin{tabular}{|c|c|}
  \hline
  \textbf{Client} & \textbf{Server} \TBstrut \\
  \hline
  With P$\in F[X]$ & \TBstrut \\
  Compute R corresponding to P & \TBstrut \\ 
  Setup(R) $\rightarrow (\sigma,\tau)$  & \TBstrut \\
  Send $\sigma$, R and the $a_i$ input to the server & \TBstrut \\
  \hline
     & The server compute R with the $a_i$ input to get the : \TBstrut \\ 
     & $ a_i$ output and the witness \TBstrut \\
     & Then he generate : \TBstrut \\
     & $Prove(R, \sigma, a_{1...l}, a_{l+1...m}) \rightarrow \pi$ \TBstrut \\
     & Send $\pi, a_i$ ouput, to the client \TBstrut \\
    \hline
    Someone who want to check the proof & \TBstrut \\ 
    run $Verify(R, \sigma, a_{i...l}, \pi) \rightarrow 0/1$ & \TBstrut \\
  \hline
\end{tabular}
\\
\\ If the ouput is 1 the client knows that the server computation is correct.

\subsubsection{Protocol in ZK}
Now we have our protocol let's imagine that the client don't wan't his polynomial P to be known by the server. So what we want is that the server don't learn anything about what he's currently computing.
\\ A possible solution is to cipher the polynomials P before creating the R1CS with paillier encryption. It's the aim of the third chapter in zkSNARK.

\newpage
\subsubsection{Summarize of the first part of zkSNARK}
\begingroup
    \fontsize{10pt}{12pt}\selectfont

\begin{tabular}{|c|c c c|}
  \hline
  & \textbf{Client} & \textbf{Communications} & \textbf{Server} \TBstrut\\
  \hline
  & With P$\in F[X]$ & & \Tstrut \\
  & Compute R corresponding to P & & \\ 
  & $\alpha, \beta, \gamma, \delta, s  \leftarrow^\$F^*$ & &\\ 
  & $\alphaOne = [\alpha]_1, \betaOne = [\beta]_1, \gammaOne = [\gamma]_1$ & & \\
  & $\deltaOne = [\delta]_1,\SOne=\{[s^i]_1\}_{i=0}^{n-1}, $  & & \\
  Setup & $\CisDeltaOne = \{[\frac{C_i (s)}{\delta}]_1\}_{i=l+1}^m$ & & \\
  & $\SAnnexe = \{[\frac{\beta A_i (s) + \alpha B_i (s)}{\delta}]_1\}_{i=l+1}^m, $  & & \\
  & $\SSecondAnnexe = \{[\frac{s^i Z_i(s)}{\delta}]_1\}_{i=0}^{n-1}$ & & \\
  & $\betaTwo = [\beta]_2,\gammaTwo = [\gamma]_2, \deltaTwo = [\delta]_2, \STwo = \{[s^i]_2\}_{i=0}^{n-1}$ & & \\
  & crs=$(\alphaOne, \betaOne,\gammaOne ,\deltaOne, \SOne, $  & & \\
  & $\CisDeltaOne,\SAnnexe, \SSecondAnnexe, \betaTwo,\gammaTwo ,\deltaTwo, \STwo, g, h, G_1, G_2)$ & $\xrightarrow[]{crs, a_{input}, R}$ & \\
  & $S=\{g^{\frac{\beta x_i(s) + \alpha y_i(s) + z_i(s)}{\gamma}}\}_{i=0}^{n}$ & & \\
  & $\CisDeltaStartOne = \{[\frac{C_i (s)}{\delta}]_1\}_{i=0}^l$ & & \\
  & $\SThirdAnnexe = \{[\frac{\beta A_i (s) + \alpha B_i (s)}{\gamma}]_1\}_{i=0}^l$ & & \\
  & $\SOneOnC=\{[C_i s^i]_1\}_{i=0}^{n-1}$ & & \\
  & vk=$(\alphaOne, \betaTwo , S, \CisDeltaStartOne, \gammaTwo, \deltaTwo, \SThirdAnnexe, \SOneOnC)$ & & \Bstrut \\
  \hline
  & & & Compute $a_{witness}, a_{output}$ from R \Tstrut \\
  & & & on $a_{input}$ \\
  & & & $r, k \xleftarrow[]{\$}F*$ \\
  & & & $\Aa =\sum_{i=0}^m a_i A_i$ \\
  & & & $\Ba=\sum_{i=0}^m a_i B_i$ \\
  Eval & & & $\Ca =\sum_{i=0}^m a_i C_i$ \\
  & & & U = $\alphaOne \Pi_{i=0}^{m} \SOne_i^{\Aa_i} (\deltaOne)^r $ \\
  & & & \VOne = $\betaOne \Pi_{i=0}^{m} \SOne_i^{\Ba_i} (\deltaOne)^k $ \\
  & & & \VTwo = $\betaTwo \Pi_{i=0}^{m} \STwo_i^{\Ba_i} (\deltaTwo)^k $ \\
  & & & We compute H(s) such that  \\
  & & & $(\sum_{i=0}^m a_i A_i(s))(\sum_{i=0}^m a_i B_i(s))$ \\
  & & & $= \sum_{i=0}^m a_i C_i(s) + H(s) Z(s)$ \\
  & & & $S=\Pi_{i=l+1}^{m}(\SAnnexe_i)^{a_i}\CisDeltaOne_i$ \\
  & & & W = $\frac{\Pi_{i=0}^n S_i (\Pi_{i=0}^n (\SSecondAnnexe_i)^{H_i}) U^k  \VOne^k}{\delta1^{rk}} $ \\
  & & $\xleftarrow[]{\pi, a_{output}}$ & $\pi=(U, W, \VTwo)$ \Bstrut \\ 
  \hline
  & $Y=\Pi_{i=0}^{l}(\SThirdAnnexe_i)^{a_i} \CisDeltaStartOne_i$ & & \Tstrut \\
  Verif &  if $e(U, \VTwo) == e(\alphaOne, \betaTwo) e(Y, \gammaTwo) e(W, \deltaTwo) $ : return 1 & & \\
  & else : return 0 & & \Bstrut \\
  \hline
\end{tabular}

\endgroup

\subsubsection{Evaluation of the polynomial}

\subsubsection{Setup function}
\underline{Setup(R) :}
\tabNormal $\alpha \xleftarrow[]{\$} F^*$
\tabNormal $\beta \xleftarrow[]{\$} F^*$
\tabNormal $\gamma \xleftarrow[]{\$} F^*$
\tabNormal $\delta \xleftarrow[]{\$} F^*$ 
\tabNormal $s \xleftarrow[]{\$} F^*$
\tabNormal $\tau = (\alpha, \beta,\gamma,\delta,s)$
\tabNormal $\alphaOne = [\alpha]_1$
\tabNormal $\betaOne = [\beta]_1$
\tabNormal $\gammaOne = [\gamma]_1$
\tabNormal $\deltaOne = [\delta]_1$
\tabNormal $\SOne=\{[s^i]_1\}_{i=0}^{n-1}$
\tabNormal $\SAnnexe = \{[\frac{\beta A_i (s) + \alpha B_i (s) + C_i (s)}{\delta}]_1\}_{i=l+1}^m$
\tabNormal $\SSecondAnnexe = \{[\frac{s^i Z_i(s)}{\delta}]_1\}_{i=0}^{n-1}$
\tabNormal $\SThirdAnnexe = \{[\frac{\beta A_i (s) + \alpha B_i (s) + C_i (s)}{\gamma}]_1\}_{i=0}^l$
\tabNormal $\sigma1=(\alphaOne, \betaOne,\gammaOne,\deltaOne, \SOne, \SAnnexe , \SSecondAnnexe,\SThirdAnnexe, g, G_1,p)$
\tabNormal $\betaTwo = [\beta]_2$
\tabNormal $\gammaTwo = [\gamma]_2$
\tabNormal $\deltaTwo = [\delta]_2$
\tabNormal $\STwo = \{[s^i]_2\}_{i=0}^{n-1}$
\tabNormal $\sigma2=(\betaTwo,\gammaTwo,\deltaTwo, \STwo, h, G_2,p)$
\tabNormal $\sigma = (\sigma1, \sigma2, A_i, B_i, C_i)$
\tabNormal return ($\sigma, \tau$)

Only $\sigma$ is sent to the person who want to prove something.
\paragraph{Prove function}
We have $a_0 = 1$ it's a constant due to the R1CS constraint. \\
\underline{Prove(crs, $a_{1..l}$, $a_{l+1..m}$) :}
\tabNormal $r \xleftarrow[]{\$}F*$
\tabNormal $k \xleftarrow[]{\$}F*$
\tabNormal $\Aa =\sum_{i=0}^m a_i A_i$
\tabNormal $\Ba=\sum_{i=0}^m a_i B_i$
\tabNormal $\Ca =\sum_{i=0}^m a_i C_i$
\tabNormal U = $\alphaOne \Pi_{i=0}^{m} \SOne_i^{\Aa_i} (\deltaOne)^r $
\tabNormal \VOne = $\betaOne \Pi_{i=0}^{m} \SOne_i^{\Ba_i} (\deltaOne)^k $
\tabNormal \VTwo = $\betaTwo \Pi_{i=0}^{m} \STwo_i^{\Ba_i} (\deltaTwo)^k $
\tabNormal We compute H(s) such that  \[ (\sum_{i=0}^m a_i A_i(s)) (\sum_{i=0}^m a_i B_i(s)) = \sum_{i=0}^m a_i C_i(s) + H(s) Z(s) \]
\tabNormal $S=\Pi_{i=l+1}^{m}(\SAnnexe_i)^{a_i} \CisDeltaOne_i$
\tabNormal W = $\frac{(\Pi_{i=0}^n S_i) (\Pi_{i=0}^n (\SSecondAnnexe_i)^{H_i}) U^k \VOne^k}{\delta1^{rk}} $
\tabNormal $\pi=(U, W, \VTwo)$ 
\tabNormal return $\pi=(U, W, \VTwo)$, $a_l$ // the one corresponding to the output.

\subsubsection{Verify function}
\underline{$Verify(R, \sigma, a_{i...l}, \pi)$ :}
\tabNormal $\Aa = \sum_{i=0}^{l} a_i A_i$
\tabNormal $\Ba=\sum_{i=0}^{l} a_i B_i$
\tabNormal $\Ca =\sum_{i=0}^{l} a_i C_i$
\tabNormal $Y=\frac{\beta \Aa (s) + \alpha \Ba (s) + \Ca (s)}{\gamma}$
\tabNormal if $UV == \alpha \beta + Y \gamma +  W\delta $ 
\tabOne return 1
\tabNormal else
\tabOne return 0 \\
See \cite{On_the_Size_of_Pairing_based_Non_interactive_Arguments} for more explanations.
\subsection{Implementation in libsnark}
The major difference is in the setup of the verification key and the provable key, where we don't send the $A_i,B_i,C_i$ but the $A_i(s),B_i(s),C_i(s)$ with the R1CS constrainsts. And for the $\alpha, \beta, \sigma,\gamma$ they are already injected in the $A_i(s),B_i(s),C_i(s)$ above.

\hl{Give more details for the implementation}
\subsection{Results with libsnark}
I've done some implementation with libsnark and here are my results compare to our protocol with pairing and paillier, for a polynomial in clear with libsnark :

\subsubsection{Time of calculation}
Time in seconds \\

\begingroup
    \fontsize{9pt}{12pt}\selectfont
\noindent\begin{tabular}{| l | S | S | S | S | S| S | S| S | S|}
\cline{2-10}
\multicolumn{1}{c|}{}& \multicolumn{3}{c|}{\textbf{Libsnark}} & \multicolumn{3}{c|}{\textbf{Paillier 1024}} & \multicolumn{3}{c|}{\textbf{Paillier 2048}\TBstrut}  \\
\hline
\textbf{Degree} & {Setup} & {Client} &  {Server}  & {Setup} & {Client} &  {Server}  & {Setup} & {Client} &  {Server \TBstrut} \\
\hline
256 & 0.1678 & 0.0249 & 0.1640 & 0.1824 & 0.0011 & 0.1638 & 0.6749 & 0.0017 & 0.2653 \\
512 & 0.2946 & 0.0261 & 0.2843 & 0.3851 & 0.0011 & 0.3165 & 1.1360 & 0.0019 & 0.5072 \\
1024 & 0.5177 & 0.0272 & 0.5551 & 0.6832 & 0.0011 & 0.6485 & 1.8827 & 0.0017 & 0.9836 \\
2048 & 0.9732 & 0.0285 & 0.9687 & 1.3459 & 0.0011 & 1.2551 & 3.8696 & 0.0017 & 1.9426 \\
4096 & 1.7896 & 0.0267 & 1.7556 & 2.7452 & 0.0011 & 2.5792 & 7.5359 & 0.0017 & 3.9480 \\
8192 & 3.0986 & 0.02684 & 3.1388 & 5.5579 & 0.0011 & 5.3739 & 14.2259 & 0.0017 & 7.4721 \\
16384 & 5.9548 & 0.0270 & 6.0528 & 10.6597 & 0.0011 & 10.4383 & 29.2171 & 0.0020 & 15.3933 \\
32768 & 10.8519 & 0.0268 & 11.3400 & 21.3708 & 0.0011 & 20.3710 & 56.6373 & 0.0017 & 30.4662\\
65536 & 20.1288 & 0.0266 & 21.5019 & 41.8292 & 0.0011 & 41.0267 & 113.1845 & 0.0017 & 61.1323 \\
131072 & 37.8404 & 0.0265 & 40.9986 & 83.8971 & 0.0011 & 82.3237 & 225.7390 & 0.0019 & 122.0703  \\
\hline
\end{tabular}
\endgroup

\subsubsection{Data to save}
Data in bits
\\

\begingroup
    \fontsize{9pt}{12pt}\selectfont
\noindent\begin{tabular}{| l | S | S | S | S | S|}
\cline{2-6}
\multicolumn{1}{c|}{}& \multicolumn{5}{c|}{\textbf{Libsnark}} \TBstrut \\
\hline
\textbf{Degree} & {Client save after setup} & {Data send to server setup} &  {Server save after setup}  & {Eval data send} & {Response of data eval \TBstrut}\\
\hline
256 & 3629 & 1482698 & 1482698 & 254 & 2548 \Tstrut\\
512 & 3629 & 2966474 & 2966474 & 254 & 2548 \\
1024 & 3629 & 5934026 & 5934026 & 254 & 2548 \\
2048 & 3629 & 11869130 & 11869130 & 254 & 2548 \\
4096 & 3629 & 23739338 & 23739338 & 254 & 2548 \\
8192 & 3629 & 47479754 & 47479754 & 254 & 2548 \\
16384 & 3629 & 94960586 & 94960586 & 254 & 2548 \\
32768 & 3629 & 189922250 & 189922250 & 254 & 2548 \\
65536 & 3629 & 379845578 & 379845578 & 254 & 2548 \\
131072 & 3629 & 759692234 & 759692234 & 254 & 2548 \Bstrut \\
\hline
\end{tabular}
\endgroup
\begingroup
    \fontsize{9pt}{12pt}\selectfont
\noindent\begin{tabular}{| l | S | S | S | S | S|}
\cline{2-6}
\multicolumn{1}{c|}{}& \multicolumn{5}{c|}{\textbf{Relic}} \TBstrut \\
\hline
\textbf{Degree} & {Client save after setup} & {Data send to server setup} &  {Server save after setup}  & {Eval data send} & {Response of data eval \TBstrut}\\
\hline
256 & 5376 & 722944 & 722944 & 256 & 768 \Tstrut\\
512 & 5376 & 1443840 & 1443840 & 256 & 768 \\
1024 & 5376 & 2885632 & 2885632 & 256 & 768 \\
2048 & 5376 & 5769216 & 5769216 & 256 & 768 \\
4096 & 5376 & 11536384 & 11536384 & 256 & 768 \\
8192 & 5376 & 23070720 & 23070720 & 256 & 768 \\
16384 & 5376 & 46139392 & 46139392 & 256 & 768 \\
32768 & 5376 & 92276736 & 92276736 & 256 & 768 \\
65536 & 5376 & 184551424 & 184551424 & 256 & 768 \\
131072 & 5376 & 369100800 & 369100800 & 256 & 768 \Bstrut \\
\hline
\end{tabular}
\endgroup
\\
\\ With our results we can see that libsnark use more memory to store necessary information for the protocol for the server side and less for the client side. But we have to take into account that our implementation of libsnark doesn't cipher our polynomial so anyone can see the polynomial we are currently evaluating. Adversely the instance of Relic cipher it.