\subsubsection{HE.ParamGen function}
\underline{Cyclotomic polynomial :}
In mathematics the n-th cyclotomic polynomial for any positive integer n is the unique irreducible polynomial with integer coefficients that is a divisor of $x^n-1$ and is not a divisor of $x^k-1$ for nay k<n. It's roots are all n-th primitive roots of unity $e^{2i\pi k/n}$. The n-th cyclotomic polynomial is equal to
\[ \Phi_m(X) = \prod_{1\leq k\leq n,  gcd(k,n) = 1} (x-e^{2i\pi k/n}) \]

Given the security parameter $\lambda$ we define the polynomial ring R = $\mathbb{Z}[X]/\Phi_m(X)$ where  $\Phi_m(X)$ is the m-th cyclotomic polynomial in $\mathbb{Z}[X]$ whose degree n = $\varphi(m)$ is lower bounded by a function of the security parameter $\lambda$. \\
The message space M is the ring $R_p = R/pR = \mathbb{Z}_p[X]/\Phi_m(X)$. The ciphertext space is describe as follow, pick an integer q > p which is co-prime to p and define the ring $R_q = R/qR = \mathbb{Z}_q[X]/\Phi_m(X)$.  \\
Ciphertexts can be thought of as polynomials in $\mathbb{Z}_q[X][Y]$ as follow :
Encryption manipulated with addition : degree 1 in Y and degree (n - 1) in X. c$\in \mathbb{Z}_q[X][Y]$ where c=$c_0 + c_1 Y$ with $c_0,c_1 \in R_q$ \\
Encryption manipulated with multiplication : degree 2 in Y and degree 2(n - 1) in X. c$\in \mathbb{Z}_q[X][Y]$ where c=$c_0 + c_1 Y + c_2 Y^2$ with $c_0,c_1,c_2 \in R_q$, $deg_X (c_i) = 2(n-1)$ \\
We define 2 distributions : \\
$D_{\mathbb{Z}^n, \sigma}$ The discrete Gaussian with parameter $\sigma$ it's a random variable over $\mathbb{Z}^n$ obtained from sampling $x\in \mathbb{R}^n$ with probability $e^{-\pi \lvert \lvert x \rvert \rvert_2 / \sigma^2}$ \\
$ZO_n$ sample a vector  x = ($x_1, ..., x_n$) with $x_i \in {-1, 0, +1}$ and Pr[$x_i=-1$] = 1/4, Pr[$x_i=+1$] = 1/4, Pr[$x_i=0$] = 1/2 \\
\underline{HE.ParamGen($\lambda$) :} 
\tabNormal $D_{\mathbb{Z}^n, \sigma}$ \\
\tabNormal $ZO_n$ \\